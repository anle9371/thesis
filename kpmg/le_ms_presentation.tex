%%%%%%%%%%%%%%%%%%%%%%%%%%%%%%%%%%%%%%%%%
% Fancyslides Presentation
% LaTeX Template
% Version 1.0 (30/6/13)
%
% This template has been downloaded from:
% http://www.LaTeXTemplates.com
%
% The Fancyslides class was created by:
% Paweł Łupkowski (pawel.lupkowski@gmail.com)
%
% License:
% CC BY-NC-SA 3.0 (http://creativecommons.org/licenses/by-nc-sa/3.0/)
%
%%%%%%%%%%%%%%%%%%%%%%%%%%%%%%%%%%%%%%%%%

%----------------------------------------------------------------------------------------
%	PACKAGES AND OTHER DOCUMENT CONFIGURATIONS
%----------------------------------------------------------------------------------------

\documentclass{fancyslides}

\usepackage[utf8]{inputenc} % Allows the usage of non-english characters
\usepackage{times} % Use the Times font
\usepackage{booktabs} % Allows the use of \toprule, \midrule and \bottomrule in tables
\graphicspath{{images/}} % Location of the slide background and figure files

% Beamer options - do not change
\usetheme{default} 
\setbeamertemplate{navigation symbols}{} % Disable the slide navigation buttons on the bottom of each slide
\setbeamercolor{structure}{fg=\yourowntexcol} % Define the color of titles and fixed text elements (e.g. bullet points)
\setbeamercolor{normal text}{fg=\yourowntexcol} % Define the color of text in the presentation

%------------------------------------------------
% COLORS
% The following colors are predefined in this class: white, black, gray, blue, green and orange

% Define your own color as follows:
\definecolor{bulgarianrose}{rgb}{0.28,0.02,0.03}
%\definecolor{rose}{rgb}{0.54,0.2,0.14}
%\definecolor{charcoal}{rgb}{0.21,0.27,0.31}
\newcommand{\structureopacity}{0.75} % Opacity (transparency) for the structure elements (boxes and circles)

\newcommand{\strcolor}{bulgarianrose} % Set the color of structure elements (boxes and circles)
\newcommand{\yourowntexcol}{white} % Set the text color

%----------------------------------------------------------------------------------------
%	TITLE SLIDE
%----------------------------------------------------------------------------------------

\newcommand{\titlephrase}{Spatially Random Processes in
  One-Dimensional Maps: The Logistic Map and The Arnold Circle Map} % Presentation title
\newcommand{\name}{Amy Le} % Presenter's name
\newcommand{\affil}{CU Boulder} % Presenter's institution
\newcommand{\email}{April 2, 2015} % Presenter's email address

\begin{document}

\startingslide % This command inserts the title slide as the first slide

%----------------------------------------------------------------------------------------
%	PRESENTATION SLIDES
%----------------------------------------------------------------------------------------

%\fbckg{1.jpg} % Slide background image
\begin{frame}
\itemized{
\item Groundwater contamination
\item Spatially Random Processes
\item Implementation in the Logistic Map and the Circle Map
}
%\pointedsl{} % Text in this environment is printed in a circle and will be made large and uppercase - if you need to fit more text in you can reduce the font size within the \pointedsl{} bracket as usual, e.g. \pointedsl{\large smaller main point}
\end{frame}
%------------------------------------------------

\begin{frame}
\misc{ % Anything can be placed inside the \misc{} command
The total groundwater withdrawls in the United States in 2005, categorized
  in terms of use.
\begin{figure}[h]
\centering
\includegraphics[scale=0.7]{gwuse}
\end{figure}
}
\end{frame}

%------------------------------------------------

\begin{frame}
\misc{ % Anything can be placed inside the \misc{} command
Typical porosity and hydraulic conductivity ranges for clay, sand, and gravel.
\begin {table}[h]
\begin{center}
    \begin{tabular}{ l l l l}
    \toprule
    Grain Size & Material & Porosity & Hydraulic Conductivity $K$
    (m/s)\\ 
\midrule
    Fine & Clay  &  50\% &$[5\times 10^{-9}, 5\times 10^{-6}]$\\ 
    Medium &  Sand & 25\% & $[10^{-8},10^{-6}]$\\ 
    Coarse &  Gravel &  20\% & $[5 \times 10^{-4}, 5 \times 10^{-2}]$\\ 
    \end{tabular}
\end{center}
\end{table}
}
\end{frame}
%------------------------------------------------
\begin{frame}
\framedsl{Spatially Random Processes} % Text in this environment will be made large, uppercase and will wrap multiple lines
\misc{
Log-normal noise in a homogeneous process
\begin{align*}
\begin{split}
\xi(x)&=\ln(R(x)) \\
\mu &= E[\xi(x)] = \ln(r)\>, \quad  r \in [0,\infty)\\
%\sigma^2 &= E[(\xi(x) - \mu)^2]=E[\xi(x)^2]-(\ln(r))^2.
\end{split}
\end{align*}
}
\end{frame}
%------------------------------------------------
\begin{frame}
%\framedsl{} 
\misc{Normal noise
\begin{align*}
\begin{split}
\xi(x)&= \ln(r) + \sum_{n \in \mathbb{Z}}\hat{\xi}_ne^{2\pi inx}\\
\hat{\xi}_{n}^* &= \hat{\xi}_{-n} \>, \quad  \hat{\xi}_n \in \mathbb{C}\\
\hat{\mu}_n&=E[\hat{\xi}_n]=0\\
\hat{\sigma}_n^2&=S(n)
\end{split}
\end{align*}
}
\end{frame}
%------------------------------------------------
\begin{frame}
\misc{
Spectral density, variance of $\hat{\xi}_n$, and covariance of $\xi(x)$
\begin{align*}
\begin{split}
\hat{\sigma}_n^2&=S(n)\\
S(n)&=\alpha e^{-L|n|}\>, \quad  L \in (0,\infty), \alpha \in \mathbb{R}\\
C(x) &= \sum_{n\in \mathbb{Z}}S(n)e^{2\pi inx}\\
&=\alpha \frac{e^{2L}-1}{e^{2L}-2\cos(2\pi x)e^L+1}\\
\end{split}
\end{align*}
}
\end{frame}
%------------------------------------------------
\begin{frame}
\misc{ 
\begin{itemize}
\item The correlation function $C(x)$
\item The spectral density $S(n)$
\end{itemize}
$L=0.5,\sigma=0.0216, \alpha=0.000114$
\begin{figure}[h]
\centering
\includegraphics[width=.48\textwidth]{cx}\hfill
\includegraphics[width=.51\textwidth]{sn}
\end{figure}
}
\end{frame}
%------------------------------------------------
\begin{frame}
\misc{ The correlation function $C(x)$ for $L \in \{0.1,0.5,1\}$. 
\begin{figure}[h]
\centering
\includegraphics[width=.33\textwidth]{correlation_L01}\hfill
\includegraphics[width=.33\textwidth]{correlation_L05}\hfill
\includegraphics[width=.33\textwidth]{correlation_L1}
\end{figure}
}
\end{frame}


%------------------------------------------------

\begin{frame}
\misc{
Choosing a distribution: uniform 
\begin{equation*}
\hat{\xi}_n \sim Unif(-M_n,M_n)
\end{equation*}
\begin{equation*}
   h_u(a_n,b_n)= \left\{
     \begin{array}{lr}
       \frac{1}{4 M_n^2} & |a_n|,|b_n| \leq M_n\\
       0 & |a_n|,|b_n| > M_n\\
     \end{array}
   \right.
\end{equation*} 
\begin{align*}
\begin{split}
\alpha &= \sigma^2 \tanh(L/2)\\
\sigma &\in \left(0,\ln\left(\frac{4}{r}\right)\sqrt{\frac{2}{3}}\frac{\tanh(L/4)}{\sqrt{\tanh(L/2)}}\right]
\end{split}
\end{align*}
}
\end{frame}
%------------------------------------------------
\begin{frame}
\misc{
Choosing a distribution: Gaussian 
\begin{equation*}
\hat{\xi}_n \sim N(0,\hat{\sigma}_n^2)
\end{equation*}
\begin{equation*}
   h_g(x) = \frac{1}{\hat{\sigma}_n \sqrt{2\pi}}e^{-\frac{x^2}{2\hat{\sigma}_n^2}}
\end{equation*} 
}
\end{frame}

%------------------------------------------------

\begin{frame}
\misc{ % Anything can be placed inside the \misc{} command
The function $R:[0,1]\to [0,4]$ where $\hat{\xi}_n \sim
Unif(-M_n,M_n), \sigma=0.0386, L=0.1, r=3.5, N=100$
\begin{figure}[h]
\centering
\includegraphics[scale=0.5]{xi}
\end{figure}
}
\end{frame}
%------------------------------------------------

\begin{frame}
\misc{ 
The function $\Omega:\mathbb{R}^+ \to \mathbb{R}^+$ where $\hat{\xi}_n \sim
N(0,\hat{\sigma}_n^2), L=0.1, r=0.7, \alpha = 10^{-6}, N=100$
\begin{figure}[h]
\centering
\includegraphics[scale=0.5]{Omega}
\end{figure}
}
\end{frame}

%------------------------------------------------

%\fbckg{2.jpg} % Slide background image
\begin{frame}
\framedsl{Logistic Map} % Text in this environment will be made large, uppercase and will wrap multiple lines
\misc{
\begin{equation*}
x_{n+1} = f_l(x_n) = rx_n(1-x_n)\>, \quad  r \in [0,4]
\end{equation*}

}
\end{frame}
%------------------------------------------------

\begin{frame}
\misc{ 
\begin{itemize}
\item Deterministic logistic map, $r=3.2$
\item A random realization, $r=3.2,L=0.1,N=100,\sigma=0.0204$ 
\end{itemize}
\begin{figure}[h]
\centering
\includegraphics[width=.495\textwidth]{det_cobweb}\hfill
\includegraphics[width=.495\textwidth]{rand_cobweb}
\end{figure}
}
\end{frame}
%------------------------------------------------

\begin{frame}
\misc{ 
Bounds on the random logistic map
\begin{itemize}
\item $\sigma=0.1093,r=2.7,L=0.9,N=112$
\item $\sigma=0.0086,r=3.5,L=0.05,N=200$
\end{itemize}
\begin{figure}[h]
\centering
\includegraphics[width=.495\textwidth]{envelope_500_r27_L09}\hfill
\includegraphics[width=.495\textwidth]{envelope_500_r35_L005}
\end{figure}
}
\end{frame}

%------------------------------------------------

%\fbckg{2.jpg} % Slide background image
\begin{frame}
\misc{ The bifurcation diagram of the deterministic logistic map
\begin{figure}[h]
\centering
\includegraphics[width=.495\textwidth]{det_bif_1}\hfill
\includegraphics[width=.495\textwidth]{det_bif_2}
\end{figure}
}
\end{frame}
%------------------------------------------------

\fbckg{rlog_bif_L_01} 
\begin{frame}
\end{frame}
%------------------------------------------------

\fbckg{rlog_bif_zoom_L_01} 
\begin{frame}
\end{frame}
%------------------------------------------------

\fbckg{rlog_bif_L_05} 
\begin{frame}
\end{frame}
%------------------------------------------------

\fbckg{rlog_bif_zoom_L_05} 
\begin{frame}
\end{frame}
%------------------------------------------------

\fbckg{rlog_bif_L_09} 
\begin{frame}
\end{frame}
%------------------------------------------------

\fbckg{rlog_bif_zoom_L_09} 
\begin{frame}
\end{frame}
%------------------------------------------------
\fbckg{blank}
\begin{frame}
\misc{ The Lyapunov exponent
\begin{itemize}
\item the deterministic map
\item $L=0.1, N = 100, x_0 = 0.7, N_\lambda  =10,000$
\end{itemize}
\begin{figure}[h]
\centering
\includegraphics[width=.495\textwidth]{det_log_lyap}\hfill
\includegraphics[width=.495\textwidth]{rlog_lyap_L01_zoom}
\end{figure}
}
\end{frame}
%------------------------------------------------
\fbckg{blank}
\begin{frame}
\misc{ The Lyapunov exponent \\
$L \in \{0.1,0.5,0.9\}, N = 100, x_0 = 0.7, N_\lambda  =10,000$
\begin{figure}[h]
\centering
\includegraphics[width=.329\textwidth]{rlog_lyap_L01_zoom}\hfill
\includegraphics[width=.329\textwidth]{rlog_lyap_L_05}\hfill
\includegraphics[width=.329\textwidth]{rlog_lyap_L_09}\\
\end{figure}
}
\end{frame}

%------------------------------------------------

\fbckg{blank}
\begin{frame}
\framedsl{Circle Map} % Text in this environment will be made large, uppercase and will wrap multiple lines
\misc{
\begin{align*}
\begin{split}
x_{n+1}= F(x_n) &=  x_n + \omega - \frac{k}{2\pi}\sin(2\pi x_n)\\
f(x_n) &= x_n + 2\pi \omega - \frac{k}{2\pi}\sin(2\pi x_n) \mod 1
\end{split}
\end{align*}
}
\end{frame}

%------------------------------------------------

\begin{frame}
\misc{ 
\begin{itemize}
\item Deterministic circle map for $k=1,\omega=0.3$
\item $\hat{\xi}_n \sim Unif(-M_n,M_n), \omega = 0.3, k=1, \alpha = 10^{-5},L=0.1, N=100$
\item $\hat{\xi}_n \sim N(0,\hat{\sigma}_n^2), \omega = 0.3, k=1, \alpha = 10^{-5},L=0.1, N=100$
\end{itemize}
\begin{figure}[h]
\centering
\includegraphics[width=.329\textwidth]{detcirc_cobweb}\hfill
\includegraphics[width=.329\textwidth]{randcirc_cobweb}\hfill
\includegraphics[width=.329\textwidth]{randcirc_norm_cobweb}\hfill
\end{figure}
}
\end{frame}
%------------------------------------------------

\begin{frame}
\misc{ % Anything can be placed inside the \misc{} command
The Arnold Tongues. This plot samples 1000 values of $\omega
  \in [0,1]$ and $k \in [0,1.5]$, and checks for periodicity up to period 100. 
\begin{figure}[h]
\centering
\includegraphics[scale=0.28]{tongues_1000_det}
\end{figure}
}
\end{frame}

%------------------------------------------------

%\fbckg{2.jpg} % Slide background image
\begin{frame}
\itemized{ % This environment simply prints a series of bullet points
\item or as a list containing multiple points
\item alternatively, you may want a few main points appearing one by one\ldots
}
\end{frame}

%------------------------------------------------

%\fbckg{2.jpg} % Slide background image
\begin{frame}
\framedsl{\pitem{First point} \pitem{Second point} \fitem{Third point}} % Text in \pitem commands will be printed one after another on separate slides until all are displayed
\end{frame}

%------------------------------------------------

%\fbckg{2.jpg} % Slide background image
\begin{frame}
\misc{ % Anything can be placed inside the \misc{} command
\Huge
Numbered list:
\begin{enumerate}
\centering
\item First item
\item Second item
\item Third item
\end{enumerate}
}
\end{frame}

%------------------------------------------------

%\fbckg{1.jpg} % Slide background image
\begin{frame}
\thankyou % Inserts a thank you slide
\end{frame}

%------------------------------------------------

% \fbckg{blank} % A blank background can be used instead of an image
% \begin{frame}
% \sources{ % An environment for giving credit for slide backgrounds, images will need to be scaled down if there are more than two
% \includegraphics[scale=0.048]{1.jpg} \ flickr/lovelornpoets\\
% \includegraphics[scale=0.2]{2.jpg} \ flickr/apsmuseum
% }
% \end{frame}

%----------------------------------------------------------------------------------------

\end{document}