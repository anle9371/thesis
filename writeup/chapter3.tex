\chapter{Results}
\section{Implementation of Randomness in the Logistic Map}
\begin{figure}[!h]
\caption[Average number of order $p$ orbits for the random circle
map]{Average number of order $p$ orbits for the random circle
map, where $r =3.2$,$L=0.1$,$N=100$, and $\sigma = 0.0061$. Results from 2000
simulations of these parameters are plotted.}
	\begin{center}
          \includegraphics[scale=0.7]{figs/rlog_hist_r32_L01.png}
	\end{center}
\end{figure}

\begin{figure}[!h]
\caption[Bifurcation diagram of the random logistic map]{Bifurcation diagram of the random
logistic map, where $r \in [0,4]$, $\Delta r = 0.008$, $N=100$, $\sigma
= 0.0061$, and $L=0.1$. Results from 500
simulations of these parameters are plotted. The colorbar to the right
of the graph demonstrates the period order and corresponding color of
the orbit.}
	\begin{center}
		\includegraphics[scale=0.7]{figs/rlog_bif_L_01_3.png}
	\end{center}
\end{figure}

\section{Implementation of Randomness in the Circle Map}
\begin{figure}[!h]
\caption[Average number of order $p$ orbits for the random circle
map]{Average number of order $p$ periodic orbits for the random circle
map, where $L=0.1$, $\omega =0.1$, $\alpha = 10^{-5}$ and $k=1$. Results from 1000
simulations of these parameters are plotted.}
	\begin{center}
		\includegraphics[scale=0.7]{figs/rcirc_avg_num_1000_sim_logscale.png}
	\end{center}
\end{figure}

\begin{figure}[!h]
\caption[Bifurcation diagram of the random
circle map]{Bifurcation diagram of the random
circle map for $L=0.1$, $\omega \in [0,1]$, $\Delta \omega = 0.001$,$\alpha = 10^{-5}$
and $k=1$. Results from 100 simulations of these parameters are
plotted. Blue: period 1, red:
period 2, cyan: period 3, magenta: period 4, green: period 5, black:
period $p > 5$.} 
	\begin{center}
		\includegraphics[scale=0.7]{figs/rcirc_bif_L01_k1.png}
	\end{center}
\end{figure}

\begin{figure}[htp]
\caption[The devil's staircase for the random circle map]{The devil's
  staircase for $k=1$,$\alpha = 10^{-5}$ and $L = 0.1,0.5$. For small $L$
  (leftmost graph), the noise is more pronounced than for large $L$
  (rightmost graph).}\label{fig:randdevil1}
\centering
\includegraphics[width=.5\textwidth]{figs/rdevil_k1_L01.png}\hfill
\includegraphics[width=.5\textwidth]{figs/rdevil_k1_L05.png}
\end{figure}

\begin{figure}[htp]
\caption[Histogram of rotation numbers in the random circle
map]{Histogram of rotation numbers in the random circle map, where
  $L=0.1$, $k=1$, $\alpha = 10^{-5}$,and $\omega = 0.45$. Results from 1000 simulations
  are plotted.}
\centering
\includegraphics[width=.5\textwidth]{figs/hist_rho_k1_L01_om045.png}\hfill
\includegraphics[width=.5\textwidth]{figs/kde_rho_k1_L01_om045.png}
\end{figure}

\begin{figure}[htp]
\caption[The Arnold tongues for the random circle map]{The Arnold
  tongues for $k\in [0,1]$, $\Delta k = 0.001$, $\omega \in [0,1]$,
  $\Delta \omega = 0.001$ and $L_j =0.1j$, where $j = 1, 2, 3, 4, 7,
  9$. Plots are ordered left to right, and top to bottom. Red: period 1
  fixed points, black: orbits that have period larger than 100 (or
  possibly no period), orange: period 2 orbit, yellow: period 3 orbit, blue: period 5 orbit,
green: period 4 orbit, and purple: period 6 orbit.}\label{fig:randtongues}
\centering
\includegraphics[width=.5\textwidth]{figs/tongues_1000_L_01.png}\hfill
\includegraphics[width=.5\textwidth]{figs/tongues_1000_L_02.png}\\
\includegraphics[width=.5\textwidth]{figs/tongues_1000_L_03.png}\hfill
\includegraphics[width=.5\textwidth]{figs/tongues_1000_L_04.png}\\
\includegraphics[width=.5\textwidth]{figs/tongues_1000_L_07.png}\hfill
\includegraphics[width=.5\textwidth]{figs/tongues_1000_L_09.png}\\
\end{figure}

\section{Dynamic Load Balancing on the Janus Supercomputer}
\begin{figure}[!h]
\caption[Load Balancing Workflow]{Load Balancing Workflow}
	\begin{center}
          \includegraphics[scale=0.5]{figs/workflow.png}
	\end{center}
\end{figure}
\begin{figure}[!h]
\caption[Load Balancing Tool Overview]{Load Balancing Tool Overview}
	\begin{center}
          \includegraphics[scale=0.45]{figs/load_balancer.png}
	\end{center}
\end{figure}

