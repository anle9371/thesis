\chapter{Conclusion}
\section{Spatially Random Processes and 1-Dimensional Maps}
This thesis explored various visualizations of bifurcation diagrams
for the logistic and circle map, but establishing a formal way of
expressing the bifurcation diagram for spatially random maps would
be advantageous for exploring higher-dimensional maps.

Between both the logistic map and the circle map, a commonality is the
question: how many period $p$ orbits is the spatially random process
responsible for? This question is another way of asking how to find
the expected number of zeros of the function $\mathcal{F}(x)=f(x)-x$, where $f(x)$ represents the
randomized map. The distribution of zeros of
$\mathcal{F}(x)$ would confer a greater understanding of how the
spatially random process stabilizes or destabilizes a map. Some prior
work in analyzing the number of zeros of a function with a random
process based on a Fourier series has been done, but in a slightly
different vein; the random process had independent and identically
distributed random variables~\cite{kahane}. We are interested in a theoretical
explanation of the probability density function of random process for
uniformly and normally distributed Fourier mode amplitudes
(independent, non-identical). Figure~\ref{fig:tmsm} suggests $\xi(x)$
is normally distributed when the Fourier modes are drawn from a
uniform distribution, but this is far from a generalization. When the
modes are normally distributed, we achieve a log-normal spatially
random process that mimics the noise in hydraulic modeling. Therefore,
the circle
map based on normally distributed modes most closely models the fluid
flow in the aquifer. Thus, the
next step is to delve into the theoretical role a log-normal process plays in the
stability of the one-dimensional map, by perhaps exploring how the
log-normal process affects the chaotic basin of attraction in the map,
or by finding the expected number of zeros of $\mathcal{F}(x)$.

\subsection{Logistic Map}
The spatially random processes appear to stabilize the logistic map in
certain regions of the bifurcation diagram, specifically where $r \in [3,4]$
(Figure~\ref{fig:rlogbif}, \ref{fig:rlogbif_zoom}). Evidence of the
newly stable regions is the presence of low-period orbits and negative
Lyapunov exponents for the random map (Figure~\ref{fig:rloglyap2}). On
the other hand, the presence of high-period orbits for
small $r$ in Figure~\ref{fig:rlogbif} and high density of positive
Lyapunov exponents (Figure~\ref{fig:rloglyap2}) suggests the possibility of chaos.

As expected, reducing the noise introduced in the logistic
map by halving the variance of the random process resulted in
bifurcation diagrams that were more similar to the deterministic
case. There were fewer high-period orbits for small values of $r$ and the spread of orbit
locations was smaller. However, despite reducing
the magnitude of the noise, the random process continues to stabilize
the map (Figure~\ref{fig:rlogbif_hs}). Negative Lyapunov exponents calculated
for this level of variance (Figure~\ref{fig:rloglyap2_hs}) support the
idea of stabilization, however, there was a high density of positive
exponents as well, suggesting chaotic tendencies.

The preservation of the negative spike in the plots of Lyapunov
exponents for the random map
(Figure~\ref{fig:rloglyap2},~\ref{fig:rloglyap2_hs}) may be explained
by the construction of the bound on $\sigma$ in~(\ref{sigma}). As $r \to 4$, the
dominant term $\ln(4/r)
\to 0$, so the standard deviation of the spatially random process diminishes as
$r$ increases. This would cause the random map to adopt more features
of the deterministic map for large $r$. An alternative is to consider constructing
another bound on $\sigma$ that does not fall off as sharply as~(\ref{sigma}).

Although the distribution of period (Figure~\ref{fig:rloghist},~\ref{fig:rloghist2})
appeared exponential, the log-scale plot of the histograms
demonstrates an exponential distribution is unlikely, since the shape
of the data is nonlinear (Figure~\ref{fig:rloghistlog}). When halving
$\sigma$, the general shape of the histogram is retained, although it is scaled a
little smaller. It appears the density of
stable orbits for $r<3$ diminishes as $L$ is increased (Figure~\ref{fig:rloghist_hs},~\ref{fig:rloghist2_hs}). Additionally,
there are fewer stable high period orbits when $\sigma$ is small. 
\subsection{Circle Map}
In contrast to the bifurcation diagrams of the logistic map, which
mostly retained the general shape of the deterministic map, the
Arnold tongues (Figure~\ref{fig:rcirctongues_u},~\ref{fig:rcirctongues_n}) of the
random circle map were asymmetric, and for some values of $L$, did not
resemble tongues at all. This suggests the randomness has an overall
destabilizing effect. Figure~\ref{fig:rcirctongues_n2} demonstrated
that the Arnold tongues for normally distributed $\hat{\xi}_n$ undergo
complicated fluctuations when $L$ is increased by $\Delta L=0.025$ in the range [0.1,0.3].

Examining the Lyapunov exponents of the map
for various values of $L$ exposed further asymmetries. Figure~\ref{fig:rcirclyap_u},~\ref{fig:rloglyap2_hs}
and Figure~\ref{fig:rcirclyap_n} showed a skewed distribution of
exponents on the right side of the plots, compared to the left
side. The mechanism for this asymmetry remains to be determined.

For uniformly distributed Fourier modes, increasing $\alpha$ seems to increase the number of
high-period orbits for certain values of $\omega < 0.2$, eliminate period
2 orbits when $\omega >0.9$, and reduce the number of period 4 orbits
in the center tongue
(Figure~\ref{fig:rcirctongues_u},~\ref{fig:rcirctongues_u_ha}). The
histograms in Figure~\ref{fig:rcirchist_u_ma1} and~\ref{fig:rcirchist_u_ha1} support this idea. For
instance, the observed frequency of
period 1 and 2 orbits have opposite trends when $\alpha$ is
halved. Also, depending on $\alpha$, one may observe period 3 and 6 orbits dominate the plot evenly, or period 3 is half as frequently observed than
period 6. Unlike the random logistic map, it does not seem that the
distribution of period is consistent in the circle map when the
variance of the Fourier modes is reduced.  

Mainly, the Arnold tongue diagrams for the circle map based on the uniform and
normal distributions were qualitatively similar, though it appeared the normal
case produced smoother plots of rotation numbers
(Figure~\ref{fig:randdevil1_n}, \ref{fig:randdevil2_n}). Perhaps the
difference between the normal and uniform cases is best highlighted in
the tongue diagrams for $L = 0.3$ (middle left of
Figure~\ref{fig:rcirctongues_u} and middle right of
Figure~\ref{fig:rcirctongues_n}). There are many high-period orbits
for $\omega\approx 0.5$ in the uniform case, but very
few in the normal case. Also, there is a period 2 tongue on the right
side of the plot in the normal case, which is absent in the uniform
case. A similarity between the simulations involving the uniform and normal
distributions is that they both produced plots of Lyapunov exponents that seem to have a
high density of exponents in the lower right corner
(Figure~\ref{fig:rcirclyap_u} and Figure~\ref{fig:rcirclyap_n}), but
only for the case where one varies $\omega$ and fixes $k$ to be
constant.

Just as the histograms in Figures~\ref{fig:rloghist},~\ref{fig:rloghist2} tried to suggest a
distribution of periods in the logistic map, kernel density estimation
in Figure~\ref{fig:kde1_u} points to the rotation numbers of the map
being normally distributed (for a large enough bandwidth) for
uniformly distributed $\hat{\xi}_n$. This notion is supported by the
results from Figure~\ref{fig:avgcircorbs} and~\ref{fig:avgcircorbs_ha}, where the average fraction of orbits was shown not to follow an exponential
distribution. One difference between the log scale plots is that for $\alpha=\frac{1}{2}10^{-5}$, the range
of average fraction of periodic orbits seems wider than for
$\alpha=10^{-5}$. 

For the normally distributed $\hat{\xi}_n$, halving
$\alpha$ introduces more stable high-period orbits when
$L=0.05,\omega=0.9,k=1$ (left side of Figure~\ref{fig:rcirchist_n1} and~\ref{fig:rcirchist_n1_ha}). Reducing
$\alpha$ also reduces the frequency of period 6 orbits for
$L=0.9,\omega=0.6,k=1.5$ (right side of Figure~\ref{fig:rcirchist_n2}
and~\ref{fig:rcirchist_n2_ha}). Studying the results from the circle
map for uniformly distributed
$\hat{\xi}_n$ was different from the normally distributed case in
terms of quantifying period distribution. First,
the observed frequency of period 1 and 2 orbits have opposite trends when $\alpha$ is
halved (Figure~\ref{fig:rcirchist_u_ma1}
and~\ref{fig:rcirchist_u_ha1}). Also, in Figure~\ref{fig:rcirchist_u_ma2},
period 3 and 6 orbits dominate the plot evenly, yet in
Figure~\ref{fig:rcirchist_u_ha2}, period 3 is half as frequently observed than
period 6. From this perspective, it does not seem that the circle map
with two different random processes is similar at all.

The numerical simulations insinuate distributions of
rotation number and period for the circle map, but they leave analytic
underpinnings to be desired. A next step would be to use the theory
surrounding the distribution of the spatially random process to derive
a more general result.
\section{Future Work}
\subsection{Extension to Higher Dimensions}
A natural and intuitive next step would be to introduce a spatially
random process to a higher dimensional set of differential equations, such as the
two-dimensional Lotka-Volterra
system for competitive species interaction
\begin{align}
\begin{split}
\dot{x} &= x(a-bx-cy),\\
\dot{y} &= y(d-ex-fy).
\end{split}
\end{align}
In this system, the parameters $a,b,c,d,e,f$ are positive
constants. A
future work could replace one of the parameters with a spatially
random process. Limit cycles arise in analyzing this system of
differential equations, and it would be interesting to explore the
effect of spatial perturbations on these cycles. The system of differential equations is more relevant to
the fluid flow problem, but another possible
choice is the two-dimensional discrete time dynamical system, the H\'{e}non map,
\begin{align}
\begin{split}
x_{n+1}&=1-ax_n^2+y_n\\
y_{n+1}&=bx_n.
\end{split}
\end{align}
One might replace one of the constants $a$ or $b$ with a spatially random
process. This map was meant to be a simplified model of the
Poincar\'{e} section of the Lorenz model. 
\subsection{Basin of Attraction for Chaotic Trajectories}
Exploring the basin of attraction for chaotic
regions of the two maps may offer some interesting insights on how
best to initialize remediation. Specifically, we would like to explore the
question: which set of initial conditions lead trajectories to chaotic
behavior instead of stable orbits? The basin of attraction may offer
implications on how one should inject treatment solution in the
aquifer to get the best (chaotic) mixing. If a formal description of
the onset of chaos for these types of systems were to exist, it would
have implications for studying higher-dimensional systems.

One way to quantify whether a trajectory is chaotic is to calculate
the Lyapunov exponent, so a future study may attempt to calculate the
probability of a positive Lyapunov exponent for any given initial
condition. The probability of a positive Lyapunov exponent could be
calculated by averaging the observed exponents for many simulations of
the random map. This calculation would be a quantitative measure that
may distinguish between the images of Lyapunov exponents from
Chapter~\ref{results}, as they are currently only qualitatively assessed.
\subsection{Stabilizing Effects of Spatial Randomness}
An unexpected result of the numerical simulations of the logistic map
was the presence of stable low-period orbits where $r \in
[3,4]$. Previously, this region was unstable for the deterministic
map. A natural question that arises is: how small can we make the
variance of the spatially random process such that this stable region is
preserved? This thesis explored using the maximum upper bound $\sigma=\sigma_{max}$ and half
of the maximum $\sigma=\frac{1}{2}\sigma_{max}$ from the expression
in~(\ref{sigma}). Looking at smaller values of $\sigma$, or choosing
an alternative method of bounding the variance would be a thought-provoking future study. The circle map also displayed stable period 2 orbits where there
were previously only period 1 orbits for $L=0.3$, but this region
disappeared for $L\leq 0.1$ and $L \geq 0.5$. It would be interesting
to find the largest $\epsilon > 0$ such that $L=0.3 \pm \epsilon$
results in the birth of this period 2 region in the circle map. Further study of the stabilized
regions in both maps may yield a theoretical
explanation for this behavior. The idea of using random processes to
stabilize chaotic behavior is a largely unexplored research topic of
high interest and potential applications. For example, Hitczenko and Medvedev derived a condition to stabilize weakly
unstable equilibria in temporally random
processes~\cite{hitczenko}. 
\subsection{Dynamic Load Balancing on the Supercomputer}
The original implementation of the numerical simulations was written
in MATLAB, and quite inefficient in certain places. We explored and simulated the spatially random logistic map using the
dynamic load balancing tool on Janus\footnote{Janus has 1368 compute nodes, and
  each node has 12 processors. Each processor is capable of
  independently carrying
  out a series of computations.}, the University of Colorado's supercomputer
\cite{janus}. The recursive nature of the map prevents the individual
calculations of an orbit from being parallelized, but a set of
iterations may be load balanced over many
cores. Improvements to this project include: adapting the program to
produce other types of plots, extending the program to model
the Arnold Circle Map, and optimizing the post-simulation data
processing. Furthermore, we can explore using Newton's method to find
the fixed points of the randomized maps. The advantages of this method
over simply iterating the maps would be quicker convergence to the
roots and identification of unstable orbits. The current iterative
method only locates stable orbits, and its rate of convergence is
highly variable. 

Appendix~\ref{lbdetails} offers more information regarding
background information on load balancing tools, program design, and effects of the load balancing tool on performance.